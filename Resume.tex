\documentclass[11pt,a4paper,roman,unicode]{moderncv}
\PassOptionsToPackage{
  pdfpagelabels=false,
  urlcolor=magenta,
  colorlinks=True,
  citecolor=gray}{hyperref} 

\usepackage{mathpazo}

\moderncvtheme[orange]{classic}                
\usepackage[utf8]{inputenc}
\usepackage[top=1.1cm, bottom=1.1cm, left=2cm, right=2cm]{geometry}

\setlength{\hintscolumnwidth}{3.8cm}
\setlength{\makecvtitlenamewidth}{12cm}
\renewcommand*{\namefont}{\fontsize{24}{29}\mdseries\upshape}

\firstname{\textit{Isaac}}
\familyname{\textit{Kamga}}
\title{M.Sc. Computer Science}   
\email{isaac.kamga@ubuea.cm}
\photo[64pt]{Candle.jpg}
\address{}
\mobile{(+237)674106297}
\homepage{https://github.com/Izakey}

\begin{document}
\maketitle

\section{Education}
\cventry{2010 -- 2015}{MSc. Computer Science}{University of Buea, Cameroon.}{}{}{\textit{Implementing a heart-shaped primitive in BRL-CAD}}
\cventry{2009 -- 2010}{Part-Qualified Accountant}{The Association of Certified Chartered Accountants}{Glasgow, United kingdom}{}{\textit{Accountant In Business(\textbf{$F_1$}), Management Accounting(\textbf{$F_2$}), Financial Accounting(\textbf{$F_3$}), Corporate And Business Law(\textbf{$F_4$}) \& Performance Management(\textbf{$F_5$})}}
\cventry{2005 -- 2008}{BSc. Mathematics and Computer Science}{University of Buea, Cameroon.}{}{}
{\textit{Magna Cum Laude}}

\section{Professional Experience}
\cventry{\llap{Oct 2012 -- Oct 2014}}{Research engineer}{PARIETAL -- INRIA, Neurospin CEA, Saclay}{}{}{
Non-smooth convex optimization; preprocessing and statistical analysis of fMRI data; registration algorithms;
machine learning on fMRI data; software engineering\newline{}}
\cventry{Sep 2011 -- Oct 2012}{Freelancer and Open-Source}{Various employers}{}{}{
Simulations for CR (Cognitive Radio) research; Windows system programming (DLLs, user-space
root-kits, etc.); implementation of Machine Learning algorithms\newline{}}

\cventry{Mar 2011 -- Aug 2011}{Cryptology and Security intern}{P1 Security}{Paris}{France}{
Implementation of an event-driven pentesting framework for telecom and VoIP-like protocols
\newline{}}

\section{Computing Skills}
\cvitem{\llap{Programming Languages}}{C/C++, Javascript, \LaTeX}
\cvitem{Software Engineering}{OOP, TDD, EDD, version control (git, github, bitbucket), continuous integration (travis)}
\cvitem{Operating Systems}{Linux, Windows (plus shell scripting \& system programming skills)}
\cvitem{Github profile}{\url{https://github.com/Izakey}}

\section{Scientific Publications (see complete google scholar)}
\cvitem{2014}{\begin{itemize}
  \item{A. ABRAHAM, E. DOHMATOB, B. THIRION, D. SAMARAS, G. VAROQUAUX,
``\emph{Region segmentation for sparse decompositions: better brain parcellations from rest fMRI}''.
\url{http://stmi2014.ece.cornell.edu/papers/STMI-P-9.pdf}}
  \item{B. THIRION, G. Varoquaux, E. DOHMATOB, J.-B. POLINE,
``\emph{Which fMRI clustering gives good brain parcellations?}''.
Frontiers in Neuroinformatics. \url{http://journal.frontiersin.org/Journal/10.3389/fnins.2014.00167/abstract}}
  \item{E. DOHMATOB, A. Gramfort, B. THIRION, G. Varoquaux
    ``\emph{Benchmarking solvers for TV-$\ell_{1}$  least-squares and logistic regression in brain imaging}''.
    Pattern Recoginition in Neuroimaging (PRNI), \emph{IEEE}. \url{http://hal.inria.fr/hal-00991743}}
\end{itemize}}
\cvitem{2013}{
\begin{itemize}
\item{A. ABRAHAM, E. DOHMATOB, B. THIRION, D. SAMARAS, and G. VAROQUAUX,
  ``\emph{Extracting brain regions from rest fMRI with Total-Variation constrained dictionary learning}''.
  MICCAI - 16th International Conference on Medical Image Computing and Computer Assisted Intervention - 2013 (2013).
  \url{http://hal.inria.fr/hal-00853242}}
\end{itemize}
}

\section{Contributions to open-source software projects}
\cvitem{Neuro-Imaging}{nipy \url{http://nipy.org}, nilearn \url{http://nilearn.github.io},
  pypreprocess \url{https://github.com/neurospin/pypreprocess}}
\cvitem{Personal projects}{See complete list on my github profile: \url{https://github.com/dohmatob}}
%% \cvitem{My \emph{Open Source Report Card}}{Tentatively, an impartial automatically generated statistical summary of
%% my ``contributions heat map'' can be found at \url{http://osrc.dfm.io/dohmatob/}}
\section{Scientific Talks}
\cvitem{PRNI 2014}{At the PRNI (Pattern Recoginition in Neuroimaging) conference that took place
3rd -- 6th June 2014 (Max-Planck Institute for Intelligent Systems, Tuebingen -- Germany), I presented my work,
``\emph{Benchmarking solvers for TV-$\ell_{1}$ least-squares and logistic regression in brain imaging}''
(\url{http://hal.inria.fr/hal-00991743})%% , under the ``Advances in fMRI analysis'' section of the conference
  .}
\cvitem{Forum STIC 2014}{Poster presentation for PRNI2014 paper at STIC,
Paris-Saclay, France.}

\cvitem{OHBM 2015}{Oral + poster presentation on\textit{``SpaceNet:
Multivariate brain decoding and segmentation''},  Honolulu, Hawaii, USA}

\cvitem{PRNI 2015}{Oral presentation on\textit{``Speeding-up
model selection in GraphNet via early-stopping and
feature-screening''}, Stanfod, USA}

\section{Hackathon Experience}
%% \cvitem{\llap{Parietal retreat,}\\ 6th -- 8th April 2014}{Virgile FRITSCH and I did VBM (Voxel-Based Morphometry) on a public dataset (Oasis database). The outcome of this
%% sprint is summarized here \url{https://github.com/Parietal-INRIA/parietal-python/wiki/VBM-dataset-for-nilearn}}
\cvitem{\llap{Google Hash Code Paris, }\\ 2014}{Implementation of
  street-viewer for Paris. Problem can be modelled as a TSP.}
\cvitem{\llap{Brainhack Paris,}\\ 23rd -- 26th Oct 2013}{Group
  analysis on Henson's multi-modal faces vs objects dataset.}

\section{Languages}
\cvline{Bilingual}{English (\textit{fluent}), French (\textit{fluent})}

\section{Awards and Scholarships}
\cvitem{2014}{Research Excellence Awards in Computer Sciece awarded at The University of Buea, Cameroon}
\cvitem{2010 -- 2013}{President of the Republic of Cameroon's Academic Excellence Awards}

\section{Interests}
\cvline{Research}{convex optimization, nonlinear registration, machine
learning, human connectome mapping, game theory}
\cvline{Hobbies}{Reading, dancing, running}

\end{document}

